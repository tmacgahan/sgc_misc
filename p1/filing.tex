\section{Electronic Filing Portal and Electronic Filing Fee Payment System}
\subsection{notes}

Pre-existing methods of accepting payments online:
\begin{itemize}
	\item PayPal offers an integration for websites
	\item Paysimple is a package that I know little about: https://paysimple.com/
	\item https://www.2checkout.com/
	\item there are certainly others
\end{itemize}

My intuition is that PayPal is going to be the most robust online payments
system.  There's an API that I don't fully understand, but assuming we can get
a token from it that says that a payment has been processed, then we can
associate that with the payment that was made, put it in the db, and do what we
like with it.  So in one sense, we can view the database as the interface with
the rest of the system.  Probably we want to keep the payments db isolated from
the rest of the system.

There are api sandboxes for testing api integration with the system.

From our perspective: there are multiple oppotunities for terrible liabilities
if we implement our own financial system.  Definitely outsource this.

\paragraph{Stated requirements} ``The Contractor shall provide a Financial
Management System which includes robust accounting functionality, to include:
an electronic fee payment system that is fully-integrated with the general
ledger, comprehensive transactional audit trail functionality, agile
calculation structures, full receipting, sophisticated debits and credits
reporting functionality, including established state-financial reporting
structures, whereby all payments and disbursements are auto-related into the
case ledger, in real time."

Thus, there is:
\begin{itemize}
	\item Financial transaction processing
	\item A general ledger
	\item Audit log
	\item Calculation system (assuming this means accounting software)
	\item Receipting (this is most certainly handled by paypal)
	\item State-financial accounting practices unification
	\item Rules for relating payments and disbursements in the ledger
\end{itemize}

\subsection{Rough design}
So it appears that there's 3 basic layers to this.

There's an interface with whatever payments system we end up using.  Whatever
that payment system doesn't really matter.  The idea is to make sure there is
some system that is unified with a database that is a canonical record of
transactions.  The mission of this system should be to answer the question "Was
this payment processed in this amount at this time" with perfect correctness
with respect to actual money that changed hands.  This system should be a
ledger system only.  It should expose a read-only authenticated API for reading
transaction data.  This transaction data should then be replicated to a
secondary system.  This replication should be secure and one-way.  Once
replicated at the target system, this target system would consitute the basis
for the financial application component of the overall delivered system.

This second layer would then contain all the services for calculation, display
of ledger, unification of state financial accounting practices, rules for
relating payments and suchlike.  The mission of this component should be to
unify the transaction data from primary to the model that is expected for
end-user interaction where the end-user is the accountant or other financial
administrator or authorized employee.

\begin{figure}[h]
	\begin{center}
		\includegraphics[scale=0.4]{images/FinancialSystem.pdf}
	\end{center}
	\caption{High Level DMS Financial System Architecture}
	\label{fig:fsArch}
\end{figure}

\subsection{Design elaboration}
Referring to figure \ref{fig:fsArch}, we see a number of components.

\paragraph{Buyer / Fee Payer} this denotes the consumer of the system that will
pay fees to the client.  This layer need not be a direct interaction with the
API, but may encompass a web page that interacts with the API, or a desktop
application on a dedicated terminal, or even a mobile application.

\paragraph{Internal Payment API} \textit{this is probably badly named.}  Even
so, this is the API that processes intention to pay a fee.  It is held to be in
communication with some external payment processing system.  It also
communicates directly with the canonical transaction database.  Its job is to
dispatch intention to pay a fee to the external payment processing system,
which in turn will reply with evidence that a payment has been processed.  If
this payment has been processed, then the canonical transaction database is
notified.

\paragraph{Canonical Transaction Database}  This is considered to be the master
authority on whether or not a transaction has taken place.  It is to store
\textbf{only} transaction records.  Once written, no record is \textbf{ever} to
be removed.  Any updates, fixups, or retractions must be entered as a separate
transaction.  Rows must all be double entry, indicating a quantity removed from
an account for every quantity deposited to an account, and these quantities
must always match.  If there is a mistake recorded in this database, it is to
\textbf{remain} and be corrected with an updating transaction.  It will respond
to no requests for any of its data, except for the two databases that are to
replicate it.

\paragraph{Mirror Transaction Database} This database is entirely for backing
up the canonical database.  If necessary or desired, this can be used to feed a
data warehouse.  It only reads from the canonical database, it cannot push to
it.

\paragraph{Ledger Application Database} This database will mirror the canonical
database.  However, it will also contain other tables for use in the ledger
application, and will ensure that relations are correctly updated for the
purposes of the ledger application.

\paragraph{Ledger Application} The ledger application will provide the
functionality required by accountants or other employees.  Here, you will need
a data scientist to appropriately design all the required functionality for
state law compliance and so forth.  This is an API, and may be interacted with
by website, dedicated terminal, pc application or mobile application, as
required.

\paragraph{Financial Administrator or Authorized Employee} This is the final
end in the role of employee of the client.

\subsection{Resources}
We estimate resources required for implementation to be in the range of the
following:

\begin{itemize}
	\item Payment API, transaction database, mirror database and external
	payment system integrations should be coverable by 2 senior level
	programmers in the space of 3-6 months.
	\item Ledger application database, ledger api should be coverable by 2
	senior level programmers, or 1 senior level programmer and one data
	scientist in the space of 3-6 months.
	\item Web application interface for buyer should require 1-2 mid level
	programmers, in the space of 3-6 months.
	\item Ledger application would require 1 senior level and 1-2 mid level
	programmers in the space of 6-12 months.  This is expected to be
	involved.
\end{itemize}
