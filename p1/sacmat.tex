\documentclass{sig-alternate-05-2015}

\usepackage{enumitem}
\usepackage{graphicx}
\usepackage{cite}
\usepackage{latexsym}
\usepackage{amssymb}
\usepackage{amsmath}
\usepackage{amsfonts}
\usepackage{algorithm}
\usepackage[noend]{algorithmic}
\usepackage{xspace}
\usepackage{listings}
\usepackage{color}
\usepackage{mdframed}

\lstset{language=C,showspaces=false,captionpos=b}

\setlist[enumerate]{label*=\arabic*.}

\begin{document}

% Copyright
\setcopyright{acmcopyright}
\CopyrightYear{2017}
%\setcopyright{rightsretained}
\conferenceinfo{SACMAT'17}{June 21-23, 2017, Indianapolis, Indiana, United States}
%\isbn{978-1-4503-3802-8/16/06}

\title{Provable Enforcement of HIPAA-Compliant Release of Medical Records Using the History Aware Programming Language}

\numberofauthors{5}
%\author{
%% 1st. author
%\alignauthor Thomas MacGahan\\
%       \affaddr{University of Texas at San Antonio}\\
%       \affaddr{thomas.macgahan@gmail.com}\\
%% 2nd. author
%\alignauthor Claiborne Johnson\\
%       \affaddr{University of Texas at San Antonio}\\
%       \affaddr{imm589@my.utsa.edu}\\
%% 3rd. author
%\alignauthor Armando Rodriguez\\
%       \affaddr{University of Texas at San Antonio}\\
%       \affaddr{armando.rodriguez@utsa.edu}\\
%\and
%% 4th. author
%\alignauthor Jeffery von Ronne\\
%       \affaddr{Google Inc.}\\
%       \affaddr{vonronne@acm.org}\\
%% 5th. author
%\alignauthor Jianwei Niu\\
%       \affaddr{University of Texas at San Antonio}\\
%       \affaddr{jianwei.niu@utsa.edu}\\
%}

\maketitle

%  Use this command to print the description
\printccsdesc

\lstset{ %
  backgroundcolor=\color{white},   % choose the background color; you must add \usepackage{color} or \usepackage{xcolor}; should come as last argument
  basicstyle=\scriptsize,          % the size of the fonts that are used for the code
  breakatwhitespace=false,         % sets if automatic breaks should only happen at whitespace
  breaklines=true,                 % sets automatic line breaking
  captionpos=b,                    % sets the caption-position to bottom
  deletekeywords={...},            % if you want to delete keywords from the given language
  escapeinside={\%*}{*)},          % if you want to add LaTeX within your code
  extendedchars=true,              % lets you use non-ASCII characters; for 8-bits encodings only, does not work with UTF-8
  frame=none,	                   % adds a frame around the code
  keepspaces=true,                 % keeps spaces in text, useful for keeping indentation of code (possibly needs columns=flexible)
  keywordstyle=\color{black},      % keyword style
  language=Octave,                 % the language of the code
  morekeywords={*,...},            % if you want to add more keywords to the set
  numbers=none,                    % where to put the line-numbers; possible values are (none, left, right)
  showspaces=false,                % show spaces everywhere adding particular underscores; it overrides 'showstringspaces'
  showstringspaces=false,          % underline spaces within strings only
  showtabs=false,                  % show tabs within strings adding particular underscores
  stepnumber=1,                    % the step between two line-numbers. If it's 1, each line will be numbered
  tabsize=2,	                   % sets default tabsize to 2 spaces
  title=\lstname                   % show the filename of files included with \lstinputlisting; also try caption instead of title
}

\begin{abstract}
Dependence on reliable information systems to safeguard personally identifiable
information implies a need for privacy policies which guide the release and
management of such information, whose mismanaged disclosure can be damaging to
both the subject and the organization that releases it.  Enforcing such
policies requires attention to detail and care, and thus any aid that a
compiler can render may be of value.  We present a demonstration of compiler
enforcement of privacy policy by implementation of the History Aware Programming
Language (HAPL) framework.  This framework allows expression of arbitrary HAPL code
for actors in an actor system to be used to back a web application.  This code is
then checked for compliance with privacy policies described in assume-guarantee
form before being assembled into a functioning application.  The framework is
demonstrated by implementing five use cases based on scenarios described in the
Health Insurance Portability and Accountability Act (HIPAA), and the performance
of the framework is tested.
%\keywords{privacy policy; assume-guarantee specifications; static analysis; scala; web applications}
\end{abstract}

%\input{introduction}
%\input{foundations}
%\input{architecture}
%\input{framework}
%\input{implementation}
%\input{examples}
%\input{conclusion}
section{main} hell world!

%\bibliographystyle{abbrv}
%\bibliography{sacmat,actors,CCS11,proposal}

\end{document}
