\documentclass{sig-alternate-05-2015}

\usepackage{enumitem}
\usepackage{graphicx}
\usepackage{cite}
\usepackage{latexsym}
\usepackage{amssymb}
\usepackage{amsmath}
\usepackage{amsfonts}
\usepackage{algorithm}
\usepackage[noend]{algorithmic}
\usepackage{xspace}
\usepackage{listings}
\usepackage{color}
\usepackage{mdframed}

\lstset{language=C,showspaces=false,captionpos=b}

\setlist[enumerate]{label*=\arabic*.}

\begin{document}

% Copyright
\setcopyright{acmcopyright}
\CopyrightYear{2017}
%\setcopyright{rightsretained}
\conferenceinfo{SACMAT'17}{June 21-23, 2017, Indianapolis, Indiana, United States}
%\isbn{978-1-4503-3802-8/16/06}

\title{Temporary}

\numberofauthors{5}
%\author{
%% 1st. author
%\alignauthor Thomas MacGahan\\
%       \affaddr{University of Texas at San Antonio}\\
%       \affaddr{thomas.macgahan@gmail.com}\\
%% 2nd. author
%\alignauthor Claiborne Johnson\\
%       \affaddr{University of Texas at San Antonio}\\
%       \affaddr{imm589@my.utsa.edu}\\
%% 3rd. author
%\alignauthor Armando Rodriguez\\
%       \affaddr{University of Texas at San Antonio}\\
%       \affaddr{armando.rodriguez@utsa.edu}\\
%\and
%% 4th. author
%\alignauthor Jeffery von Ronne\\
%       \affaddr{Google Inc.}\\
%       \affaddr{vonronne@acm.org}\\
%% 5th. author
%\alignauthor Jianwei Niu\\
%       \affaddr{University of Texas at San Antonio}\\
%       \affaddr{jianwei.niu@utsa.edu}\\
%}

\maketitle

%  Use this command to print the description
\printccsdesc

\lstset{ %
  backgroundcolor=\color{white},   % choose the background color; you must add \usepackage{color} or \usepackage{xcolor}; should come as last argument
  basicstyle=\scriptsize,          % the size of the fonts that are used for the code
  breakatwhitespace=false,         % sets if automatic breaks should only happen at whitespace
  breaklines=true,                 % sets automatic line breaking
  captionpos=b,                    % sets the caption-position to bottom
  deletekeywords={...},            % if you want to delete keywords from the given language
  escapeinside={\%*}{*)},          % if you want to add LaTeX within your code
  extendedchars=true,              % lets you use non-ASCII characters; for 8-bits encodings only, does not work with UTF-8
  frame=none,	                   % adds a frame around the code
  keepspaces=true,                 % keeps spaces in text, useful for keeping indentation of code (possibly needs columns=flexible)
  keywordstyle=\color{black},      % keyword style
  language=Octave,                 % the language of the code
  morekeywords={*,...},            % if you want to add more keywords to the set
  numbers=none,                    % where to put the line-numbers; possible values are (none, left, right)
  showspaces=false,                % show spaces everywhere adding particular underscores; it overrides 'showstringspaces'
  showstringspaces=false,          % underline spaces within strings only
  showtabs=false,                  % show tabs within strings adding particular underscores
  stepnumber=1,                    % the step between two line-numbers. If it's 1, each line will be numbered
  tabsize=2,	                   % sets default tabsize to 2 spaces
  title=\lstname                   % show the filename of files included with \lstinputlisting; also try caption instead of title
}

\begin{abstract}
%Dependence on reliable information systems to safeguard personally identifiable
%information implies a need for privacy policies which guide the release and
%management of such information, whose mismanaged disclosure can be damaging to
%both the subject and the organization that releases it.  Enforcing such
%policies requires attention to detail and care, and thus any aid that a
%compiler can render may be of value.  We present a demonstration of compiler
%enforcement of privacy policy by implementation of the History Aware Programming
%Language (HAPL) framework.  This framework allows expression of arbitrary HAPL code
%for actors in an actor system to be used to back a web application.  This code is
%then checked for compliance with privacy policies described in assume-guarantee
%form before being assembled into a functioning application.  The framework is
%demonstrated by implementing five use cases based on scenarios described in the
%Health Insurance Portability and Accountability Act (HIPAA), and the performance
%of the framework is tested.
%\keywords{privacy policy; assume-guarantee specifications; static analysis; scala; web applications}
\end{abstract}

\section{Management}
hi to you!
\section{Document Management System}

\subsection{Requirements}
The Contractor shall provide a Document Management System which includes the
ability to house electronically-filed court documents and auto-relate them to a
case index, the ability to scan images and auto-relate them to a case index,
the ability to apply digital annotations to imaged documents as part of agile
workflow, the ability to auto-post Orders created from the Judicial Dashboard
into the case index in real-time, and to keyword search and redact elements
inside digital images housed in the Document Management System.

\subsection{Breakdown}
\begin{itemize}
	\item Storage of electronic court documents
	\item indexing of court documents to a case, presumably with a
	pre-existing id
	\item ability to scan images
	\item automatically relate scans to case index
	\item apply digital annotations to imaged documents
	\item application of digital annotations must be part of
	agile\footnote{unclear what is meant by this}
	\item push orders from Judicial Dashboard into the case index
	\item keyword search
	\item redact elements of the scans
\end{itemize}

\subsection{Necessarily Implied By Requirements}
The requirement here seems to be an indexing system that needs to plug into an
existing id.  This sounds like an SQL DB tracking files in some kind of data
store.  Presumably there needs to be a way to track and apply edits, which will
imply some kind of frontend for interpreting edits.

\paragraph{Redactions}There also appears to be a request for some kind of
editor to ``apply redactions."  Generally, if redactions are required, we
should expect this to be firm and probably important.  This has security
implications.  How do we do a redaction that truly honors the requirement that
the information so redacted is actually inaccessible?  It seems to me that a
document server that serves this will need to store an original and apply
redactions \textbf{before} the file leaves the server.  This further implies
that the system will need some kind of authorization and some kind of access
level tracking.  It will need to be verifiably secure, as well.

\paragraph{Annotations}The requirement that digital annotations be applied to
an imaged document seems like a common-ish concept.  This clearly requires both
frontend support and backend support.  You will need a method for applying
annotations, and a method for capturing these annotations and associating them
with a particular digital document.  Typical annotations systems will tend to
offer a text box, which really boils down to a top-left position for the start
of the box, and a width.  The rest of the box would be implied by the amount of
text written.  This would probably constitute a simple version, and later
versions (if necessary) could then perhaps talk about such things as font and
point size.  This is only if we implement this ourselves, although given the
specific nature of system it seems like the sort of thing that would need to be
custom built.

\paragraph{Relation to Case Index}A number of these requirements allude to
association with a case index.  Presumeably this already exists, since it
sounds like something that the court is using to track cases that exist outside
the system and not mearly some nicety of an existing system.  However, this
charge is unclear, since it could mean anything from enter a case id into the
system to get an AI system to figure out what case ID is appropriate to ``gee
you have this case id open, why don't I associate this document you're adding
to this case id?"  Of the three possibilities, I consider the last possibility
to be the most likely.

In any case, it means at the very least that redactions, annotations, and
images must all refer to a case index, and that seems to be the central idea
they are trying to get across.

\paragraph{Keyword Search}The way this is phrased it sounds like what they want
is a way to keyword search scanned documents.  This would need to be an OCR
system.  Otherwise, if they are just trying to keyword search metadata and
annotations, this would be trivial.  Maybe this is indeed what they want?  In
any case, there are some IR packages out there that probably aren't too bad.  I
mean how, can you really do worse than MS Outlook?

\paragraph{Document Management System}It's not entirely clear what this is.  Is
this meant to be a service?  Or an application?  Or both?  My guess would be
both.

\subsection{Proposed Solution}
etc

\section{Electronic Filing Portal and Electronic Filing Fee Payment System}
\subsection{notes}

Pre-existing methods of accepting payments online:
\begin{itemize}
	\item PayPal offers an integration for websites
	\item Paysimple is a package that I know little about: https://paysimple.com/
	\item https://www.2checkout.com/
	\item there are certainly others
\end{itemize}

My intuition is that PayPal is going to be the most robust online payments
system.  There's an API that I don't fully understand, but assuming we can get
a token from it that says that a payment has been processed, then we can
associate that with the payment that was made, put it in the db, and do what we
like with it.  So in one sense, we can view the database as the interface with
the rest of the system.  Probably we want to keep the payments db isolated from
the rest of the system.

There are api sandboxes for testing api integration with the system.

From our perspective: there are multiple oppotunities for terrible liabilities
if we implement our own financial system.  Definitely outsource this.

\paragraph{Stated requirements} ``The Contractor shall provide a Financial
Management System which includes robust accounting functionality, to include:
an electronic fee payment system that is fully-integrated with the general
ledger, comprehensive transactional audit trail functionality, agile
calculation structures, full receipting, sophisticated debits and credits
reporting functionality, including established state-financial reporting
structures, whereby all payments and disbursements are auto-related into the
case ledger, in real time."

Thus, there is:
\begin{itemize}
	\item Financial transaction processing
	\item A general ledger
	\item Audit log
	\item Calculation system (assuming this means accounting software)
	\item Receipting (this is most certainly handled by paypal)
	\item State-financial accounting practices unification
	\item Rules for relating payments and disbursements in the ledger
\end{itemize}

\subsection{Rough design}
So it appears that there's 3 basic layers to this.

There's an interface with whatever payments system we end up using.  Whatever
that payment system doesn't really matter.  The idea is to make sure there is
some system that is unified with a database that is a canonical record of
transactions.  The mission of this system should be to answer the question "Was
this payment processed in this amount at this time" with perfect correctness
with respect to actual money that changed hands.  This system should be a
ledger system only.  It should expose a read-only authenticated API for reading
transaction data.  This transaction data should then be replicated to a
secondary system.  This replication should be secure and one-way.  Once
replicated at the target system, this target system would consitute the basis
for the financial application component of the overall delivered system.

This second layer would then contain all the services for calculation, display
of ledger, unification of state financial accounting practices, rules for
relating payments and suchlike.  The mission of this component should be to
unify the transaction data from primary to the model that is expected for
end-user interaction where the end-user is the accountant or other financial
administrator or authorized employee.

\begin{figure}[h]
	\begin{center}
		\includegraphics[scale=0.4]{images/FinancialSystem.pdf}
	\end{center}
	\caption{High Level DMS Financial System Architecture}
	\label{fig:fsArch}
\end{figure}

\subsection{Design elaboration}
Referring to figure \ref{fig:fsArch}, we see a number of components.

\paragraph{Buyer / Fee Payer} this denotes the consumer of the system that will
pay fees to the client.  This layer need not be a direct interaction with the
API, but may encompass a web page that interacts with the API, or a desktop
application on a dedicated terminal, or even a mobile application.

\paragraph{Internal Payment API} \textit{this is probably badly named.}  Even
so, this is the API that processes intention to pay a fee.  It is held to be in
communication with some external payment processing system.  It also
communicates directly with the canonical transaction database.  Its job is to
dispatch intention to pay a fee to the external payment processing system,
which in turn will reply with evidence that a payment has been processed.  If
this payment has been processed, then the canonical transaction database is
notified.

\paragraph{Canonical Transaction Database}  This is considered to be the master
authority on whether or not a transaction has taken place.  It is to store
\textbf{only} transaction records.  Once written, no record is \textbf{ever} to
be removed.  Any updates, fixups, or retractions must be entered as a separate
transaction.  Rows must all be double entry, indicating a quantity removed from
an account for every quantity deposited to an account, and these quantities
must always match.  If there is a mistake recorded in this database, it is to
\textbf{remain} and be corrected with an updating transaction.  It will respond
to no requests for any of its data, except for the two databases that are to
replicate it.

\paragraph{Mirror Transaction Database} This database is entirely for backing
up the canonical database.  If necessary or desired, this can be used to feed a
data warehouse.  It only reads from the canonical database, it cannot push to
it.

\paragraph{Ledger Application Database} This database will mirror the canonical
database.  However, it will also contain other tables for use in the ledger
application, and will ensure that relations are correctly updated for the
purposes of the ledger application.

\paragraph{Ledger Application} The ledger application will provide the
functionality required by accountants or other employees.  Here, you will need
a data scientist to appropriately design all the required functionality for
state law compliance and so forth.  This is an API, and may be interacted with
by website, dedicated terminal, pc application or mobile application, as
required.

\paragraph{Financial Administrator or Authorized Employee} This is the final
end in the role of employee of the client.

\subsection{Resources}
We estimate resources required for implementation to be in the range of the
following:

\begin{itemize}
	\item Payment API, transaction database, mirror database and external
	payment system integrations should be coverable by 2 senior level
	programmers in the space of 3-6 months.
	\item Ledger application database, ledger api should be coverable by 2
	senior level programmers, or 1 senior level programmer and one data
	scientist in the space of 3-6 months.
	\item Web application interface for buyer should require 1-2 mid level
	programmers, in the space of 3-6 months.
	\item Ledger application would require 1 senior level and 1-2 mid level
	programmers in the space of 6-12 months.  This is expected to be
	involved.
\end{itemize}

section{main} hello world!

%\bibliographystyle{abbrv}
%\bibliography{sacmat,actors,CCS11,proposal}

\end{document}
